\documentclass{article}
\usepackage[utf8]{inputenc}

\begin{document}
\section{Probability Theory Review}
\subsection{Probabilistic Independence}
\begin{itemize}
	\item[a)]
	With rolling the dice twice you have 36 different outcomes. Event A gets fullfilled in 3 cases. 12 cases fullfil event B and C. Event D gets fullfilled in 15 cases. And 22 cases for Event E.
	This results in following Propabilties for the different events:
	$$P(A) = \frac{3}{36}=\frac{1}{12}$$$$P(B) =\frac{12}{36}=\frac{1}{3}$$$$ P(C) = \frac{12}{36}=\frac{1}{3}$$$$P(D) = \frac{15}{36}=\frac{5}{12}$$$$P(E) = \frac{22}{36}=\frac{11}{18}$$ 
	\item[b)] Event A and B get satisfied in the cases (6,5),(5,6),(6,6) to show dependece we have to show following equation.
	$$P(A \cap B) \neq P(A)P(B)$$
	$$\frac{3}{36} \neq \frac{3}{36}*\frac{1}{12}$$
	since both sites are not equal we know that A is dependent of event B.
	\item[c)]
	$A\cap C$ can not be fullfilled. And since $P(A)>0$ and $P(C)>0$ we know that
	$$P(A\cap C) = 0 \neq P(A)P(C)>0$$
	Which shows that A is depented of event C.
	\item[d)]
	$D\cap E$ gets satisfied by the cases (1,2),(2,3),(3,4),(4,5),(5,6) which implies $P(D\cap E)=\frac{5}{36}$
	Since $$P(D\cap E) = \frac{5}{36} \neq \frac{55}{216}=\frac{5}{12}*\frac{11}{18} = P(D)P(E)$$ we know that D and E are dependend.
\end{itemize}
\end{document}